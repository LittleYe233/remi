% Quick start guide
\documentclass{beamer}

\usepackage{xeCJK}[CJKmath = true]
\usepackage{blindtext}
\usepackage{unicode-math}
\usepackage{xcolor}
\usepackage{soul}
\usepackage{biblatex}
\usepackage[inkscapeformat=png]{svg}

\addbibresource{slides.bib}

\setmainfont{Noto Serif CJK SC}
\setsansfont[BoldFont = Fira Sans SemiBold]{Fira Sans Book}
\setmonofont{Ligalex Mono}
\setCJKmonofont{Maple Mono SC NF}
\setmathfont{Latin Modern Math}
\setCJKmainfont{KaiTi}
\setCJKsansfont{KaiTi}
\setCJKfallbackfamilyfont{\CJKrmdefault}{Noto Serif CJK SC Regular}

\usetheme{Boadilla}

\newcommand{\thiss}{}
\newcommand{\thisss}{}

\definecolor{VeryLightPink}{HTML}{FFF8EB}
\definecolor{ReddishOrange}{HTML}{FC5A26}
\sethlcolor{VeryLightPink}
\newcommand{\code}[1]{\textcolor{ReddishOrange}{\hl{\texttt{#1}}}}

\hypersetup{
    colorlinks=true,
    linkcolor=cyan,
    filecolor=cyan,
    urlcolor=cyan
}

% Title page details
\title{Pop Music Transformer: Beat-based Modeling and Generation of Expressive Pop Piano Compositions}
\subtitle{Some self-explanation and exploration}
\author{LittleYe233}
\institute{LittleYe233 Workgroup}
\date{\today}

\begin{document}

\begin{frame}
    % Print the title page as the first slide
    \titlepage
\end{frame}

\begin{frame}{Outline}
    \tableofcontents
\end{frame}

% Current section
\AtBeginSection[]
{
    \begin{frame}{Outline}
        \tableofcontents[currentsection]
    \end{frame}
}

\renewcommand{\thiss}{About the paper}
\renewcommand{\thisss}{Basic information}
\section{\thiss}
\subsection{\thisss}
\begin{frame}{\thiss}{\thisss}
    \textbf{TITLE:} \href{https://arxiv.org/abs/2002.00212}{Pop Music Transformer: Beat-based Modeling and Generation of Expressive Pop Piano Compositions}

    \textbf{AUTHORS:}
    \begin{itemize}
        \item Yu-Siang Huang <\href{mailto:yshuang@ailabs.tw}{yshuang@ailabs.tw}> from Taiwan AI Labs \& Academia Sinica, Taipei, Taiwan
        \item Yi-Hsuan Yang <\href{mailto:yhyang@ailabs.tw}{yhyang@ailabs.tw}> from Taiwan AI Labs \& Academia Sinica, Taipei, Taiwan
    \end{itemize}
    
    \textbf{KEYWORDS:} Automatic music composition, transformer, neural sequence model

    \textbf{OPEN-SOURCED CODE:} \href{https://arxiv.org/abs/2002.00212}{GitHub:YatingMusic/remi}
\end{frame}

\renewcommand{\thisss}{Research background}
\section{\thiss}
\subsection{\thisss}
\begin{frame}{\thiss}{\thisss}
    
\end{frame}

\renewcommand{\thiss}{Basic music concepts}
\section{\thiss}
\begin{frame}[allowframebreaks]{\thiss}
    \begin{itemize}
        \item \textbf{chord} - any harmonic set of pitches/frequencies consisting of multiple notes (also called "pitches") that are heard as if sounding simultaneously \cite{chordwiki}
            \begin{itemize}
                \item \textbf{bass note} - the root/lowest note in many cases \cite{bassnotewiki}
                \item \textbf{slash chord} - a chord whose bass note or inversion is indicated by the addition of a slash and the letter of the bass note after the root note letter \cite{slashchordwiki}
                \item \textbf{quality}: is similar to the "type" of a chord (i.e. major ("maj"), minor ("min"), augmented ("aug"), diminished ("dim"))
            \end{itemize}

        \framebreak

        \item \textbf{chroma feature} - a powerful tool for analyzing music whose pitches can be meaningfully categorized \cite{chromawiki}
    \end{itemize}

    \begin{center}
        \begin{figure}
            \includegraphics[width=.6\textwidth]{../assets/ChromaFeatureCmajorScaleScoreAudioColor.png}
            \caption{An example of chroma feature figure \cite{chromawiki}}
        \end{figure}
    \end{center}

    \framebreak
\end{frame}

\renewcommand{\thiss}{Analysis of source code}
\renewcommand{\thisss}{\texttt{utils.py} - utilities to convert MIDI/words to REMI}
\section{\thiss}
\subsection{\thisss}
\begin{frame}{\thiss}{\thisss}
    \textbf{Data types:}
    \begin{itemize}
        \item \code{Item} - a direct representation of note and tempo events, etc.
        \item \code{Event} - a representation of REMI events
    \end{itemize}
\end{frame}

\begin{frame}{\thiss}{\thisss}
    \textbf{Functions:}
    \begin{itemize}
        \item \code{read\_items(file\_path) -> (note\_items, tempo\_items)} - read Items from MIDI
        \item \code{quantize\_items(items, ticks=120) -> items} - align positions of Items with tick \textbf{(possible changing real time gaps between original items)}
        \item \code{extract\_chords(items) -> output} - extract chord Items from note Items
        \item \code{group\_items(items, max\_time, ticks\_per\_bar=DEFAULT\_RESOLUTION*4) -> groups} - group given Items by bar (a default bar consists of 16 quantized bins mentioned above and in the paper)
        \item \code{item2event(groups) -> events} - convert groups to REMI Events
    \end{itemize}
\end{frame}

\renewcommand{\thisss}{\texttt{chord\_recognition.py} - recognize and extract chords from notes}
\subsection{\thisss}
\begin{frame}{\thiss}{\thisss}
    \begin{center}
        \begin{figure}
            \includesvg[height=6cm]{../assets/chord_recognition_flowchart_extract.svg}
            \hspace{2cm}
            \includesvg[height=6cm]{../assets/chord_recognition_flowchart_find_chord.svg}
            \caption{Flowcharts of methods \code{extract()}'s and \code{find\_chord()}'s behavior}
        \end{figure}
    \end{center}
\end{frame}

\renewcommand{\thisss}{\texttt{modules.py} - necessary modules for TensorFlow}
\subsection{\thisss}
\begin{frame}{\thiss}{\thisss}
    
\end{frame}

\renewcommand{\thisss}{\texttt{model.py} - the Pop Music Transformer model}
\subsection{\thisss}
\begin{frame}{\thiss}{\thisss}
    
\end{frame}

\renewcommand{\thiss}{References}
\section{\thiss}
\begin{frame}[allowframebreaks]{\thiss}
    \printbibliography    
\end{frame}

\end{document}
